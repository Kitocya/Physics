\documentclass[a4paper,11pt]{ltjsarticle}
\usepackage{nomal_preamble}

\title{物理学概論Ⅰ\ 解説}
\author{ほうじ茶}
\date{}

\begin{document}

\pagestyle{fancy}
\lhead{\ 初等数学}
\rhead{}
\cfoot{\thepage}

\maketitle
\tableofcontents
\clearpage

\section{初等数学}

物理学においての最低限の数学を扱う.また,本資料は講義資料に則して作成をする.

\subsection{微分・積分}

教養科目「解析学」に該当する.古典物理学においては,微分が大きな効力を持つが運動方程式を解くうえで積分が必要となる.
なぜなら,微分と積分は表裏一体の関係であり,一般に逆の操作であるといわれている.\\

難しい議論をするならば,以下のように言える.

\begin{screen}
  \begin{center}
    \textbf{微分}\  局所的なふるまいを調べる.(傾斜)\\
    \textbf{積分}\  大域的なふるまいを調べる.(面積)
  \end{center}
\end{screen}

\hs{0}

難しいけど,簡単に言えば$1$秒後は誰でも推察できる.座っていれば座っているだろうし,
電車に乗っていれば電車に乗っているだろう.これが「微分」の基本的な考え方である.

これを何回も繰り返し行い,$1$分,$1$時間,$1$日,$1$ヶ月,$1$年へと進展させたものが
「積分」の基本的な考え方である.厳密な話をすれば間違っているが,ざっとこのようなイメージでよい.

\subsubsection{具体的な解き方}

\hs{0}

基本的には,函数と一対一対応で覚えていけば問題ない.

\begin{itemize}
  \item $(x^{n})^{\prime}=nx^{n-1}$
  \item $(\sin{\theta})^{\prime}=\cos{\theta}$
  \item $(\cos{\theta})^{\prime}=-\sin{\theta}$
  \item $(e^{x})^{\prime}=e^{x}$
  \item $(f(x)+g(x))^{\prime}=f^{\prime}(x)+g^{\prime}(x)$
  \item $(f(x)\times g(x))^{\prime}=f^{\prime}(x)g(x)+f(x)g^{\prime}(x)$
  \item $f^{\prime}(g(x))=f^{\prime}(g(x))+g^{\prime}(x)$
\end{itemize}

数式だけの議論を列挙してみたが,分かりずらいので$1$つずつ説明していく.

\clearpage

\subsubsection{$(x^{n})^{\prime}=nx^{n-1}$}

\hs{0}

「微分を定義から求めろ」っていう問いが出る可能性もあるので,なぜこの式がこの公式になるのか導出しておく.

まず,定義としてはあるグラフ上の$2$点を取り,片方の点をもう一方に限りなく近づける.
数式で議論すると,$\dfrac{\Delta y}{\Delta x}$の$\Delta x$を限りなく$0$に近づけることである.

\hs{0}

一般的に数式で記載すると,

\begin{equation*}
  f^{\prime}(x)=\displaystyle{\lim_{\Delta x \to 0} \frac{f(x+\Delta x)-f(x)}{f(\Delta x)}}
\end{equation*}

簡単に導出をできるが,別に暗記しておけばよいだけなので無視しても良い.

\begin{equation*}
  (x+\Delta x)^{n} = x^n + {}_nC_1 x^{n − 1}\Delta x + {}_nC_2 x^{n − 2}\Delta x^2 + \cdots + \Delta x^n
\end{equation*}

\begin{equation*}
  \begin{aligned}
    (x^{n})^{\prime} &=\ \displaystyle{\lim_{\Delta x \to 0} \frac{f(x+\Delta x)-f(x)}{f(\Delta x)}} \\
                     &=\ \displaystyle{\lim_{\Delta x \to 0} \frac{(x^n + {}_nC_1 x^{n−1}\Delta x + {}_nC_2 x^{n−2}\Delta x^2 + \cdots + \Delta x^n) − x^n}{\Delta x}} \\
                     &=\ \displaystyle{\lim_{\Delta x \to 0} \frac{{}_nC_1 x^{n−1}\Delta x + {}_nC_2x^{n−2}\Delta x^2 + \cdots + \Delta x^n}{\Delta x}} \\
                     &=\ \displaystyle{\lim_{\Delta x \to 0} ({}_nC_1x^{n−1} + {}_nC_2 x^{n−2}\Delta x + \cdots + \Delta x^{n−1})} \\
                     &=\ \displaystyle{{}_nC_1 x^{n−1}} \\
                     &=\ \displaystyle{nx^{n−1}}
  \end{aligned}
\end{equation*}

この定義に代入して求めることは,非常に大変なのでそこから導出された公式を用いる.
公式を適応する方法の一例としてSTEPごとに説明する.

\hs{0}

\textbf{STEP1}\ 函数$f(x)=x^{n}$の,指数$n$を底$x$の前に持ってくる.\\
\hs{60} $\longrightarrow\ f(x)={\color{cyan} n}x^{n}$

\hs{0}

\textbf{STEP2}\ 函数$f(x)=nx^{n}$の,指数$n$に$-1$する.\\
\hs{60} $\longrightarrow\ f^{\prime}(x)=nx^{n{\color{cyan} -1}}$

\hs{0}

三角函数($\sin$,$\cos$)や指数函数($e^{x}$)に関しては,そのまま覚えればよい.

\clearpage

\subsubsection{$(f(x)\times g(x))^{\prime}=f^{\prime}(x)g(x)+f(x)g^{\prime}(x)$}

\hs{0}

$2$つの数式のかたまりが見えているときに有効.

\begin{center}
  $(\text{前微分})(\text{後残し})+(\text{前残し})(\text{後微分})$
\end{center}

\subsubsection{$f^{\prime}(g(x))=f^{\prime}(g(x))+g^{\prime}(x)$}

\hs{0}

$1$つの数式のべき乗の形の時に有効.

\begin{center}
  $(\text{全体微分}) \times (\text{中身微分})$
\end{center}

\subsubsection{積分}

\hs{0}

積分は冒頭でも記載したが,微分の逆操作を行えばよい.ただ,積分を行うと解が$1$つに定まらない.微分で定数項が消えてしまうので,任意定数を導入して微分したときに支障がないようにする.また,任意定数に値を入れたものを,特殊解と呼ぶ.公式を適応する方法の一例としてSTEPごとに説明する.

\hs{0}

\textbf{STEP1}\ 函数$f(x)=x^{n}$の,指数$n$に$+1$する.\\
\hs{60} $\longrightarrow\ f(x)=x^{n{\color{cyan} +1}}$

\hs{0}

\textbf{STEP2}\ 函数$f(x)=x^{n+1}$の逆数を,底$x$の前に持ってくる.\\
\hs{60} $\longrightarrow\ \displaystyle{\int f(x)\ dx={\color{cyan} \dfrac{1}{\ n+1\ }}\ x^{n+1}}$

\subsubsection{記法について}

\hs{0}

今まで利用してきた$x^{\prime}$・$x^{\prime \prime}$(プライム)は,ラグランジュが作った記法である.他の記法は以下の表にまとめた.縦列の記法はすべて同じことを示している.

\hs{0}

\begin{table}[hb]
  \centering
  \begin{tabular}{ccc}
    \hline
      作った人     & 1階微分          & 2階微分 \\
    \hline
      ラグランジュ & $x^{\prime}$     & $x^{\prime \prime}$ \\
      ニュートン   & $\dot{x}$        & $\ddot{x}$ \\
      ライプニッツ & $\dfrac{dy}{dx}$ & $\dfrac{d^2y}{dx^2}$ \\
    \hline
  \end{tabular}
\end{table}

\clearpage

\subsubsection{微分方程式}

\hs{0}

今まで,微分を単体で行ってきたが方程式にしたらというお話.方程式は$x$を求める$1$つの値であったのに対し,微分方程式は$f(x)$という数式を求める.
基本的には,方程式の中に$f^{\prime}(x)$が含まれている.そのため,$f^{\prime}(x)=$の形になるように移項し$^\prime$が消えるように計算する.$^\prime$を消すためには,微分してある状態のものを戻す作業なので,積分を行えばよい.このあと扱うが,運動方程式は,この微分方程式を解くことによって公式を得ることができる.

\ast\ 講義資料では,「いろいろな関数の微分を試せばよい」という力技が書いてあるが,時間の無駄なので順当に方程式を解くべきだと個人的には思う.

\hs{0}

\textbf{例題}\ 微分方程式$f^{\prime}(x)-3x=0$の解を求めよ.

\begin{equation*}
  \begin{aligned}
                   \ & f^{\prime}(x)-3x      &=&\ 0 \\
    \Leftrightarrow\ & \hs{25} f^{\prime}(x) &=&\ 3x \\
    \Leftrightarrow\ & \hs{27} f(x)          &=&\ \dfrac{1 \times 3}{\ 1+1\ }\ x^{1+1} \\
    \Leftrightarrow\ & \hs{27} f(x)          &=&\ \dfrac{3}{\ 2\ }\ x^{2}
  \end{aligned}
\end{equation*}

\subsection{第2回\ 問題演習}

\begin{enumerate}
  \item 変数$x$の関数$f(x)=(2x+1)^2-5$がある.
  \begin{enumerate}
    \item[①] 微分係数$f^{\prime}(1)$を,微分の定義を使って求めよ.
    \begin{equation*}
      \begin{aligned}
        f^{\prime}(1) = \lim_{\Delta x \to 0}\dfrac{\Delta f}{\Delta x} &= \lim_{\Delta x \to 0}\dfrac{f(1+\Delta x)-f(1)}{\Delta x}\\
                                                                        &=\ \lim_{\Delta x \to 0}\dfrac{((2(1+\Delta x)+1)^{2}-5)-((2 \times 1)^{2}-5)}{\Delta x}\\
                                                                        &=\ \lim_{\Delta x \to 0}\dfrac{12 \Delta x+4 \Delta x^{2}}{\Delta x}\\
                                                                        &=\ \lim_{\Delta x \to 0}12+4 \Delta x\\
                                                                        &=\ 12+4 \times 0 \\
                                                                        &=\ 12
      \end{aligned}
    \end{equation*}
    \item[②] 導関数$f^{\prime}(x)$と微分係数$f^{\prime}(1)$を公式$(x^{n})^{\prime}=nx^{n-1}$を使って求めよ.
    \begin{equation*}
      \begin{aligned}
        f^{\prime}(x) &=\ 2(2x+1) \times (2x+1)^{\prime} \\
                      &=\ (4x+2) \times 2 \\
                      &=\ 8x+4 \\
        f^{\prime}(1) &=\ 8 \times 1 + 4 \\
                      &=\ 12
      \end{aligned}
    \end{equation*}
  \end{enumerate}
  \item 微分して$f(x)=x^{2}+2x$となる関数$F(x)$を$1$つ答えよ.\\(この問いは微分方程式だよね.)
  \begin{equation*}
    \begin{aligned}
                f(x)=\ & F^{\prime}(x) &=&\ x^{2}+2x \\
      \Leftrightarrow\ & \hs{2} F(x)   &=&\ \dfrac{1}{\ 2+1\ }\ x^{2+1}+\dfrac{2}{\ 1+1\ }\ x^{1+1} \\
      \Leftrightarrow\ & \hs{2} F(x)   &=&\ \dfrac{1}{\ 3\ }\ x^{3}+x^{2}+C \\
    \end{aligned}
  \end{equation*}

  \item 次の定積分$\displaystyle{\int_0^3(x^{2}+2x)dx}$の値を求めよ.\\
  (物理学において,定積分を用いることは滅多にないので説明は割愛する.)
  \begin{equation*}
    \begin{aligned}
      \int_0^3(x^{2}+2x)dx &=\ \left[\dfrac{1}{\ 3\ }\ x^{3}+x^{2} \right]_0^3 \\
                           &=\ \left(\dfrac{1}{\ 3\ } \cdot 3^{3} + 3^{2}\right)-\left(\dfrac{1}{\ 3\ } \cdot 0^{3} + 0^{2}\right) \\
                           &=\ 18-0 \\
                           &=\ 18
    \end{aligned}
  \end{equation*}
  \item 変数$t$の関数$x(t)=3e^{-2t}+1$がある.この関数の導関数$\dot{x}(t)$と微分係数$\dot{x}(0)$を答えよ.
  \begin{equation*}
    \begin{aligned}
          \dot{x}(t)=\ & x^{\prime}(t) &=&\ (3e^{-2t}+1t^{0})^{\prime} \\
      \Leftrightarrow\ & \dot{x}(t)    &=&\ 3e^{-2t} \times (-2) +1 \times 0 \\
      \Leftrightarrow\ & \dot{x}(t)    &=&\ -6e^{-2t} \\
                       & \dot{x}(0)    &=&\ -6e^{-2 \times 0} \\
      \Leftrightarrow\ & \dot{x}(0)    &=&\ -6e \times 1 \\
      \Leftrightarrow\ & \dot{x}(0)    &=&\ -6
    \end{aligned}
  \end{equation*}
\end{enumerate}

\subsection{第4回\ 問題演習}

\begin{enumerate}
  \item $f^{\prime}(x)+6x^{2}=0$
\begin{equation*}
  \begin{aligned}
                   \ & f^{\prime}(x)+6x^{2}  &=&\ 0 \\
    \Leftrightarrow\ & \hs{30} f^{\prime}(x) &=&\ -6x^{2} \\
    \Leftrightarrow\ & \hs{32} f(x)          &=&\ -\dfrac{6 \times 3}{\ 2+1\ }\ x^{2+1} \\
    \Leftrightarrow\ & \hs{32} f(x)          &=&\ -6\ x^{3}+C
  \end{aligned}
\end{equation*}

\clearpage

  \item $f^{\prime \prime}(x)-3x^{2}+x-7=0$
\begin{equation*}
  \begin{aligned}
                   \ & f^{\prime \prime}(x)-3x^{2}+x-7 &=&\ 0 \\
    \Leftrightarrow\ & \hs{70} f^{\prime \prime}(x)    &=&\ 3x^{2}-x+7 \\
    \Leftrightarrow\ & \hs{72} f^{\prime}(x)           &=&\ \dfrac{3 \times 1}{\ 2+1\ }\ x^{2+1}-\dfrac{1}{\ 1+1\ }\ x^{1+1}+\dfrac{7 \times 1}{\ 0+1\ }x^{0+1} \\
    \Leftrightarrow\ & \hs{72} f^{\prime}(x)           &=&\ x^{3}-\dfrac{1}{\ 2\ }\ x^{2}+7x \\
    \Leftrightarrow\ & \hs{74} f(x)                    &=&\ \dfrac{1 \times 1}{\ 3+1\ }\ x^{\ 3+1\ }-\dfrac{1 \times 1}{\ 2 \times (2+1)\ }\ x^{2+1}+\dfrac{7 \times 1} {\ 1+1\ }\ x^{1+1} \\
    \Leftrightarrow\ & \hs{74} f(x)                    &=&\ \dfrac{1}{\ 4\ }\ x^{4}-\dfrac{1}{\ 6\ }\ x^{3}+\dfrac{7}{\ 2\ }\ x^{2}+C_{1}x+C_{2}
  \end{aligned}
\end{equation*}
  \item $f^{\prime \prime}(x)-7=0$
\begin{equation*}
  \begin{aligned}
                   \ & f^{\prime \prime}(x)-7        &=&\ 0 \\
    \Leftrightarrow\ & \hs{20} f^{\prime \prime}(x)  &=&\ 7 \\
    \Leftrightarrow\ & \hs{22} f^{\prime}(x)         &=&\ \dfrac{7 \times 1}{\ 0+1\ }x^{0+1} \\
    \Leftrightarrow\ & \hs{22} f^{\prime}(x)         &=&\ 7x+C_{1} \\
    \Leftrightarrow\ & \hs{24} f(x)                  &=&\ \dfrac{7 \times 1}{\ 1+1\ }\ x^{1+1}+\dfrac{1 \times 1}{\ 0+1\ }\ C_{1}x^{0+1} \\
    \Leftrightarrow\ & \hs{24} f(x)                  &=&\ \dfrac{7}{\ 2\ }\ x^{2}+C_{1}x+C_{2}
  \end{aligned}
\end{equation*}
  \item $f^{\prime \prime}(x)=0$
\begin{equation*}
  \begin{aligned}
    \ & f^{\prime \prime}(x) &=&\ 0x^{0} \\
    \Leftrightarrow\ & f^{\prime}(x) &=&\ \dfrac{1}{\ 0+1\ }x^{0+1} \\
    \Leftrightarrow\ & f^{\prime}(x) &=&\ C_{1} \\
    \Leftrightarrow\ & f(x) &=&\ \dfrac{1 \times 1}{\ 0+1\ }\ C_{1}x^{0+1} \\
    \Leftrightarrow\ & f(x) &=&\ C_{1}x+C_{2}\ \cdots (\ast)
  \end{aligned}
\end{equation*}
  \item 4.の$f(2)=1$,$f^{\prime}(2)=3$の特殊解
\begin{equation*}
  \begin{aligned}
    \begin{cases}
      f(2)=1 \\
      f^{\prime}(2)=3
    \end{cases}
    &\Leftrightarrow
    \begin{cases}
      C_{1} \cdot 2+C_{2}=1 \\
      C_{1}=3
    \end{cases} \\
    &\Leftrightarrow \hs{11} 3 \cdot 2+C_{2}=1 \\
    &\Leftrightarrow \hs{11} 6+C_{2}=1 \\
    &\Leftrightarrow \hs{11} C_{2}=1-6 \\
    &\Leftrightarrow \hs{11} C_{2}=-5
  \end{aligned}
\end{equation*}

\begin{center}
  $C_{1}$,$C_{2}$を$(\ast)$に代入して,$3x-5$と分かる.
\end{center}

\end{enumerate}

\clearpage

\subsection{三角函数}

三角形のみで考えてみると,以下のようになる.$\sin$,$\cos$の定義は,右下の通りになり
それぞれの頭文字$S$,$C$の筆記体を書く順に分母$\rightarrow$分子となる.

\hs{0}

\begin{center}
  \begin{tikzpicture}[scale=0.5,samples=300]

    \node at (0,0) {
    \begin{tikzpicture}[scale=0.8,samples=300]
      \coordinate (A) at (5,3) {};
      \coordinate (B) at (0,0) {};
      \coordinate (C) at (5,0) {};
      \coordinate (BC) at ($(B)!0.5!(C)$);
    \node [below] at (BC) {$a$};
      \coordinate (CA) at ($(C)!0.5!(A)$);
    \node [right] at (CA) {$b$};
      \coordinate (AB) at ($(A)!0.5!(B)$);
    \node [above] at (AB) {$c$};
    \draw (A) -- (B) -- (C) -- cycle;
      \pic["$\theta$",draw,thick,angle radius=7mm,angle eccentricity=1.3,cyan] {angle = C--B--A};
      \draw [thick,cyan] (-0.024,0) -- (5.01,0);
      \pic [draw,angle radius=4mm] {right angle=A--C--B};
    \end{tikzpicture}
    };
    
    \node at (13,0) {$
        \begin{cases}
          \displaystyle{\sin{\theta}=\frac{\ b\ }{\ c\ }} &\Leftrightarrow b = c \sin{\theta}\\
          \\
          \displaystyle{\cos{\theta}=\frac{\ a\ }{\ c\ }} &\Leftrightarrow a = c \cos{\theta}\\
        \end{cases}
    $};

\end{tikzpicture}
\end{center}

これをグラフ上に表すと,下記の通りになる.

\begin{center}
\begin{tikzpicture}[scale=1.5,samples=300]
    \node at (0,0) {
      \begin{tikzpicture}[scale=1.5,samples=300]
        \draw[-Stealth,thick,cyan] (-1.6,0) -- (1.6,0) node[right] {$x$};
        \draw[-Stealth,thick] (0,-1.6) -- (0,1.6) node[above] {$y$};
          \coordinate (A) at ({1/sqrt(2)},{1/sqrt(2)}) {};
          \coordinate (B) at (0,0) {};
          \coordinate (C) at (1,0) {};
        \draw (A) -- (B);
          \pic["$\theta$",draw,angle radius=5mm,angle eccentricity=1.3,cyan] {angle = C--B--A};
        \draw (0,0) circle[radius=1];
          \node [left] at (0,{1/sqrt(2)}) {$\sin{\theta}$};
        \draw [dashed] (0,{1/sqrt(2)}) -- (A);
          \node [below,cyan] at ({1/sqrt(2)},0) {$\cos{\theta}$};
          \node [below left] at (0,0) {$\mathrm{O}$};
          \node [above right] at (1,0) {$1$};
          \node [above right] at (0,1) {$1$};
          \node [above left] at (-1,0) {$-1$};
          \node [below right] at (0,-1) {$-1$};
        \draw [dashed] (A) -- ({1/sqrt(2)},0);
        \fill ({1/sqrt(2)},{1/sqrt(2)}) circle[radius=1.2pt];
          \node [above right] at ({1/sqrt(2)},{1/sqrt(2)}) {($\cos \theta$,$\sin \theta$)};
      \end{tikzpicture}
    };
\end{tikzpicture}
\end{center}

水色で色を付けているので分かると思うが,定義から$\theta$に接している辺が$\cos \theta$倍となる.

\clearpage

\subsection{ベクトル}

簡単なイメージとしては,原っぱにいろいろな向きや長さの矢印が落ちている感じ.
「向き」いわゆる方向と,「長さ」いわゆる大きさを持つ量のことである.一般にベクトル量と呼ばれる.
今まで扱ってきたものは,「大きさ」しか持たないスカラー量というものである.

\subsubsection{ベクトルの加減法}

\hs{0}

言葉で説明するなら,矢印の終点と始点を合わせて始点と終点を線で結ぶ.減法の場合は,矢印を逆(逆ベクトル)にしてから足す.
もう1つ始点を揃えて平行四辺形という解き方もあり,運動方程式を解く際の分力を求めるのに最適である.

\hs{0}

\begin{center}
\begin{tikzpicture}
  \draw [-Stealth] (0,0)--(1,2);
  \node [above left] at ($(0,0)!0.5!(1,2)$) {$\mathbb{b}$};
  \draw [-Stealth] (1,0)--(4,0);
  \node [below] at ($(1,0)!0.5!(4,0)$) {$\mathbb{a}$};
  \draw [-Stealth] (8,0)--(9,2);
  \node [right] at ($(8,0)!0.5!(9,2)$) {$\mathbb{b}$};
  \draw [-Stealth] (5,0)--(8,0);
  \node [below] at ($(5,0)!0.5!(8,0)$) {$\mathbb{a}$};
  \draw [-Stealth,thick,cyan] (5,0)--(9,2);
  \node [above left,cyan] at ($(5,0)!0.5!(9,2)$) {$\mathbb{a}+\mathbb{b}$};
  \draw [-Stealth] (10,0)--(11,2);
  \node [above left] at ($(10,0)!0.5!(11,2)$) {$\mathbb{b}$};
  \draw [-Stealth] (10,0)--(13,0);
  \node [below] at ($(10,0)!0.5!(13,0)$) {$\mathbb{a}$};
  \draw [dashed] (11,2)--(14,2)--(13,0);
  \draw [-Stealth,thick,cyan] (10,0)--(14,2);
  \node [below right,cyan] at ($(10,0)!0.5!(14,2)$) {$\mathbb{a}+\mathbb{b}$};
\end{tikzpicture}
\end{center}

\subsubsection{ベクトルの乗法}

\hs{0}

加減法では「向き」をそのままで計算できた.しかし,乗法では向きを揃える必要がある.
以下で仕事の話をするが,物理学において仕事をするとは一方向に対して最初と最後にいた地点の差
で仕事をしているのか否かを判断する.このようにベクトルの乗法,すなまち内積(ドット積,スカラー積とも呼ばれる.)
は仕事の計算の時に用いる.

\hs{0}

\begin{center}
  \begin{tikzpicture}
    \coordinate (A) at (1,2) {};
    \coordinate (B) at (0,0) {};
    \coordinate (C) at (3,0) {};
    \coordinate (D) at (1,0) {};
    \draw [-Stealth] (B)--(A);
    \node [above left] at ($(B)!0.5!(A)$) {$\mathbb{b}$};
    \draw [-Stealth] (B)--(C);
    \node [above] at ($(0,0)!0.5!(C)$) {$\mathbb{a}$};
    \pic["$\theta$",draw,angle radius=5mm,angle eccentricity=1.3] {angle = C--B--A};
    \draw [dashed] (A)--(D);
    \draw [-Stealth,thick,cyan] (B)--(D);
    \node [below,cyan] at ($(B)!0.5!(D)$) {$\mathbb{b} \cos \theta$};
\end{tikzpicture}
\end{center}

内積はスカラー量(向きを持たない)なので大きさを表す記号$|x|$(絶対値)に入れて,

\begin{equation*}
  \mathbb{a} \cdot \mathbb{b} = |a||b|\cos \theta
\end{equation*}

乗法の記号は一般に$\times$(cross)を用いるが,内積は$\cdot$(dot)を用いる.
また,成分同士の掛け算の和でも求められる.

\clearpage

\subsubsection{単位ベクトル}

\hs{0}

ユークリッド空間(実数を$n$個並べた全体の集合)において,3つの直交座標をそれぞれ$x$軸,$y$軸,$z$軸とする.
そのなかで大きさを「1」に仕立てたベクトルを単位ベクトルという.
(単位○○は基本的に,○○の大きさを「1」に仕立てたもののことである.)

また,$x$軸と平行な単位ベクトルを$\mathbb{i}$,$y$軸と平行な単位ベクトルを$\mathbb{j}$,$z$軸と平行な単位ベクトルを$\mathbb{k}$とする.

\subsubsection{ベクトルの成分表示}

\hs{0}

単位ベクトルと係数倍を用いて,一般にベクトルを以下のような式で表せる.

\begin{equation*}
  \mathbb{a}=A \mathbb{i}+B \mathbb{j}+C \mathbb{k}
\end{equation*}

また,係数を座標のように表して,

\begin{equation*}
  \mathbb{a}=(A,B,C)
\end{equation*}

とも表せる.

\subsection{第1回\ 問題演習}

\begin{center}
\begin{tikzpicture}[scale=0.8]
  \draw[step=1,gray,very thin] (-3,-1) grid (6,6);
  \draw[-Stealth,thick] (-3,0) --(6,0);
  \draw[-Stealth,thick] (0,-1) --(0,6);
  \node[below] at (6,0) {$x$};
  \node[left] at (0,6) {$y$}; 
  \draw[-Stealth,very thick] (-1,2)--(-2,4);
  \node[above right] at ($(-1,2)!0.5!(-2,4)$) {$\mathbb{b}$};
  \draw[-Stealth,very thick] (1,5)--(1,1);
  \node[right] at ($(1,5)!0.5!(1,1)$) {$\mathbb{c}$};
  \draw[-Stealth,very thick] (2,1)--(5,3);
  \node[below right] at ($(2,1)!0.5!(5,3)$) {$\mathbb{a}$};
\end{tikzpicture}
\end{center}

\clearpage

\begin{enumerate}
  \item ベクトル$\mathbb{i}$,$\mathbb{j}$,$\mathbb{a+b}$を図示しなさい.

  \begin{center}
    \begin{tikzpicture}[scale=0.8]
      \draw[step=1,gray,very thin] (-1,0) grid (6,6); 
      \draw[-Stealth,very thick] (5,3)--(4,5);
      \node[above right] at ($(5,3)!0.5!(4,5)$) {$\mathbb{b}$};
      \draw[-Stealth,very thick] (2,1)--(5,3);
      \node[below right] at ($(2,1)!0.5!(5,3)$) {$\mathbb{a}$};
      \draw[-Stealth,very thick,cyan] (0,5)--(1,5);
      \draw[-Stealth,very thick,cyan] (2,1)--(4,5);
      \node[above left,cyan] at ($(2,1)!0.5!(4,5)$) {$\mathbb{a+b}$};
      \node[below,cyan] at ($(0,5)!0.5!(1,5)$) {$\mathbb{i}$};
      \draw[-Stealth,very thick,cyan] (0,2)--(0,3);
      \node[left,cyan] at ($(0,2)!0.5!(0,3)$) {$\mathbb{j}$};
    \end{tikzpicture}
  \end{center}

  \item ベクトル$\mathbb{a}$,$\mathbb{b}$,$\mathbb{c}$,$\mathbb{a+b}$,$\mathbb{i}$,$\mathbb{j}$の成分表示を答えなさい.

  \begin{equation*}
    \begin{aligned}
      \mathbb{a} &= (3,2) & \mathbb{a+b} &= (2,4) \\
      \mathbb{b} &= (-1,2) & \mathbb{i} &= (1,0) \\
      \mathbb{c} &= (0,-4) & \mathbb{j} &= (0,1)
    \end{aligned}
  \end{equation*}

  \item スカラー積(内積)$\mathbb{a \cdot a}$と$\mathbb{a \cdot b}$を答えなさい.

  \begin{equation*}
    \begin{aligned}
      \mathbb{a \cdot a}  &= 3 \cdot 3 + 2 \cdot 2 \\
                          &= 9+4 \\
                          &= 13 \\
      \mathbb{a \cdot b} &= 3 \cdot (-1) + 2 \cdot 2 \\
                          &= -3+4 \\
                          &= 1
    \end{aligned}
  \end{equation*}

  \item ベクトル$\mathbb{a}$の大きさを答えなさい.
  
  \begin{equation*}
    \begin{aligned}
      |\mathbb{a}| &= \sqrt{\mathbb{a \cdot a}\ } \\
                    &= \sqrt{3 \cdot 3 + 2 \cdot 2\ } \\
                    &= \sqrt{9+4\ } \\
                    &= \sqrt{13\ }
    \end{aligned}
  \end{equation*}

  \item ベクトル$\mathbb{V}_{0}$の大きさは$28$である.$\mathbb{V}_{0}$を成分表示せよ.
  
  \begin{center}
    \begin{tikzpicture}
      \coordinate (A) at (2,1.5) {};
      \coordinate (B) at (0,0) {};
      \coordinate (C) at (3,0) {};
      \coordinate (D) at (1,0) {};
      \coordinate (E) at (0,2) {};
      \draw [-Stealth] (B)--(A);
      \node [above right] at (A) {$\mathbb{V}_{0}$};
      \draw [-Stealth] (B)--(C);
      \node [below] at (C) {$x$};
      \draw [-Stealth] (B)--(E);
      \node [left] at (E) {$y$};
      \pic["$30^{\circ}$",draw,angle radius=5mm,angle eccentricity=1.3] {angle = C--B--A};
  \end{tikzpicture}
  \end{center}

\clearpage

$\mathbb{V}_0$の成分表示は三角函数を用いて,

  \begin{equation*}
    \begin{aligned}
      \mathbb{V}_0 &= (\mathbb{V}_0 \cos{30^{\circ}},\mathbb{V}_0 \sin{30^{\circ}}) \\
                    &= \left(28 \times \dfrac{\sqrt{3\ }}{2},28 \times \dfrac{1}{2}\right) \\
                    &= (14 \sqrt{3\ },14)
    \end{aligned}
  \end{equation*}  

  \item ベクトル$\mathbb{G}$の大きさは$G$である.$\mathbb{G}$を成分表示せよ.

    \begin{figure}[htbp]
      \centering
      \begin{tabular}{cc}
        \begin{minipage}[t]{0.3\linewidth}
          \begin{tikzpicture}[rotate=233]
            \coordinate (A) at (2,1.5) {};
            \coordinate (B) at (0,0) {};
            \coordinate (C) at (3,0) {};
            \coordinate (D) at (1,0) {};
            \coordinate (E) at (0,2) {};
            \draw [-Stealth] (B)--(A);
            \node [below] at (A) {$\mathbb{G}$};
            \draw [-Stealth] (B)--(C);
            \node [below] at (C) {$x$};
            \draw [-Stealth] (E)--(B);
            \node [left] at (B) {$y$};
            \pic["$\theta$",draw,angle radius=5mm,angle eccentricity=1.3] {angle = A--B--E};
        \end{tikzpicture}
        \end{minipage} &
        \begin{minipage}[t]{0.3\linewidth}
          \begin{tikzpicture}[rotate=0]
            \coordinate (A) at (2,1.5) {};
            \coordinate (B) at (0,0) {};
            \coordinate (C) at (3,0) {};
            \coordinate (D) at (1,0) {};
            \coordinate (E) at (0,2) {};
            \draw [-Stealth] (B)--(A);
            \node [above right] at (A) {$\mathbb{G}$};
            \draw [-Stealth] (B)--(C);
            \node [below] at (C) {$x$};
            \draw [-Stealth] (E)--(B);
            \node [below] at (B) {$y$};
            \pic["$\theta$",draw,angle radius=5mm,angle eccentricity=1.3] {angle = A--B--E};
          \end{tikzpicture}
        \end{minipage}
      \end{tabular}
  \end{figure}

$y$軸が上下逆であることを考慮すると,

\begin{equation*}
  \mathbb{G}=(G \sin{\theta},-G \cos{\theta})
\end{equation*}

\end{enumerate}

\clearpage

\pagestyle{fancy}
\lhead{\ 力学入門}
\rhead{}
\cfoot{\thepage}

\section{力学入門}

物理学概論で行う内容はニュートン力学という17世紀以前の天文学から派生して創られた,
物体の運動を記述する学問形態でケプラーらの観測を元にニュートンが創り上げた理論を扱う.

一方で化学概論で行っている量子力学は,古典力学が完成した(19世紀)あとに技術の進歩によって
これまでの物理学では説明できない現象(ミクロな世界)を扱っていく.これが現在の生命化学の進歩につながった.
実際に講義を受けていく中で,徐々に物理や数学的な内容が増えてきているのは,元が物理であり,数学だからである.

\subsection{位置,速度,加速度}

位置$x$は 速度$v$によって決まり,速度$v$も何かしらの影響により決まる.
これを$x$,$v$のように文字で置こうとして加速度$a$を導入された.
そのように考えると,位置$x$を集めると速度$v$になり,速度$v$を集めると加速度$a$となる.
この考え方は積分に値する.
そのため,位置$x$を基準として速度$v$を$\dot{x}$,加速度$a$を$\ddot{x}$と表せる.

\subsection{ベクトル・スカラー}

少し前に簡単に記載したが,計算方法をここで記載しておく.
簡単な違いは「向き」をもつか否かであった.
計算することのみを考えるとベクトルは「向き」(座標表記),
スカラーは「大きさ」(数値表記)だと思えばよい.

\hs{0}

\begin{table}[hb]
  \centering
  \begin{tabular}{|c|cc|}
    \hline
                   & $2$次元 & $3$次元 \xs{-10}{0}{20} \\
    \hline
      ベクトル      & $\mathbb{a}=(x,y)$ & $\mathbb{a}=(x,y,z)$ \xs{-10}{0}{20} \\
      スカラー      & $|\mathbb{a}|=\sqrt{x^{2}+y^{2}\ }$ & $|\mathbb{a}|=\sqrt{x^{2}+y^{2}+z^{2}\ }$ \xs{-10}{0}{20} \\
    \hline
  \end{tabular}
\end{table}

一般に,「速度」はベクトル,「速さ」はスカラーと一文字の違いで表すものが異なるので注意する必要がある.
この違いは物理学ができるようになるのか否かの分岐点になりうるのでしっかりと押さえておくこと.
あとは,どの単位とどの単位が等しいのか,どの矢印とどの矢印が釣り合っているのか,
その物体の運動を想像し図示できれば,力学は容易に理解できる.

\clearpage

\subsection{第3回\ 問題演習}

  \begin{enumerate}
    \item 適当なデカルト座標系を用いて,ある質点の位置ベクトル$\mathbb{r}$が,時刻$t$の関数として,以下の様に求まっている.ここで,長さと時間の単位はメートル[m]と秒[s]である.
    \begin{equation*}
      \mathbb{r}(t)=(2t+1,1,-5t^{2}+10t+3)
    \end{equation*}
    \begin{itemize}
      \item 時刻$t=0$[s]での質点の位置と原点からの距離を答えよ.
      \begin{equation*}
        \begin{aligned}
          位置\ \mathbb{r} &= ((2 \times 0)+1,1,(-5 \times 0)^{2}+(10 \times 0)+3) \\
                            &= (0+1,1,0^{2}+0+3) \\
                            &= (1,1,3) \\
          距離\ |\mathbb{r}(0)| &= \sqrt{1^{2}+1^{2}+3^{2}\ } \\
                                &= \sqrt{1+1+9\ } \\
                                &= \sqrt{11\ }
        \end{aligned}
      \end{equation*}
      \item 時刻$t=2$[s]での質点の速度と速さを答えよ.
      \begin{equation*}
        \begin{aligned}
          速度\ \dot{\mathbb{r}}(t) &= ((2t+1)^{\prime},(1)^{\prime},(-5t^{2}+10t+3)^{\prime}) \\
                            &= (2,0,-10t+10) \\
         速度\ \dot{\mathbb{r}}(2) &= (2,0,-10) \\
          速さ\ |\dot{\mathbb{r}}(2)| &= \sqrt{2^{2}+0^{2}+(-10)^{2}\ } \\
                                &= \sqrt{4+0+100\ } \\
                                &= \sqrt{104\ }
        \end{aligned}
      \end{equation*}
      \item 時刻$t=3$[s]と$t=5$[s]での質点の加速度を答えよ.
      \begin{equation*}
        \begin{aligned}
          加速度\ \ddot{\mathbb{r}}(t) &= ((2)^{\prime},(0)^{\prime},(-10t+10)^{\prime}) \\
                            &= (0,0,-10) \\
          加速度\ \ddot{\mathbb{r}}(3) &= (0,0,-10) \\
          加速度\ \ddot{\mathbb{r}}(5) &= (0,0,-10) \\
        \end{aligned}
      \end{equation*}
      \item 質点の$z$座標が最大となる時刻とその時の位置を答えよ.
      \begin{equation*}
        \begin{aligned}
          z(t)  &= -5t^{2}+10t+3 \\
                &= -5(t-1)^{2}+8\ \text{(平方完成)} \\
                &\therefore t=1\ のとき\ z_{\mathrm{max}}=8 \\
          \mathbb{r}(1) &=(2t+1,1,-5t^{2}+10t+3) \ \cdots (\ast) \\
                        &=((2 \times 1)+1,1,(-5 \times 1)^{2}+(10 \times 1)+3) \\
                        &=(3,1,8) \\
        \end{aligned}
      \end{equation*}
      \item 別解(最高点のとき速度$0$になることを利用.)
      \begin{equation*}
        \begin{aligned}
          \dot{\mathbb{r}}(t)  &= 0\\
                    \dot{z}(t) &= -10t+10 =0 \\
                               &\Leftrightarrow t=1\ \cdots あとは\ (\ast)\ の計算へ移動  \\
        \end{aligned}
      \end{equation*}
    \end{itemize}
  \end{enumerate}

\subsection{運動方程式}

世の中の物体を考えると基本的には移動だけである.
移動を考える中で物体の大きさを考えると複雑化してしまうので,便宜上一点のみを考える.
この点を質点と呼ばれる.

ニュートン力学が完成したころは,コンピュータ技術はない.そのために人間が容易に思いつくものしか数式に現れない.
しかも物体の運動は,数学のように証明ができない(経験則).
この経験則を元に,力学には法則が3つあるので以下に記載しておく.

\hs{0}

\begin{itembox}[l]{Newtonの運動\ 第1法則(慣性の法則)}
  すべての物体は,外部から力を加えられない限り,静止している物体は静止し続け,運動している物体は同じ速度で同じ方向に運動し続ける.
\end{itembox}

$\Longrightarrow$「物体が外力を受けていなければ物体は運動しない.」

\hs{0}

\begin{itembox}[l]{Newtonの運動\ 第2法則(運動方程式)}
  物体の加速度は,物体に働く力に比例し,その質量に反比例する.
\end{itembox}

$\Longrightarrow$「物体の加速度は$a=\dfrac{F}{m}$と記述できる.」

\hs{0}

\begin{itembox}[l]{Newtonの運動\ 第3法則(作用・反作用の法則)}
  2つの物体が互いに力を及ぼし合うとき,それらの力の大きさは等しく,向きは反対方向に働く.
\end{itembox}

$\Longrightarrow$「物体に力を加えると,同じ力だけ物体が押し返してくる.」

\subsubsection{運動方程式の立て方}

\vs{13}

\begin{enumerate}
  \item 物体の運動を図示する.
  \item 物体間の力(垂直抗力を忘れずに)と重力のベクトルを描き入れる.
  \item どのベクトルが釣り合っているのか否か調べ,釣り合っていないベクトルを$F$として$m\ddot{x}=F$に代入する.
  \item 立てた運動方程式を微分方程式とみなし計算する.
\end{enumerate}

\clearpage

\subsubsection{運動方程式を解くと公式になる}

\hs{0}

よく講義内で「へんな公式」という言葉が多用されている.しかし,運動方程式を解くと一般に知られている
公式になるので共有しておく.
問題を解くうえでは,運動方程式を立てた後に公式代入し同値変換したかのように見せるのが楽だろう.

\begin{equation*}
  \begin{aligned}
                    &\ m \ddot{x} = F \\
    \Leftrightarrow &\ \ddot{x} = \dfrac{F}{\ m\ } \\
    \Leftrightarrow &\ \dot{x} = C_{1}t + C_{2}\ \cdots\ (\ast) \\
    \Leftrightarrow &\ x = \dfrac{1}{\ 2\ }C_{1}t^{2} + C_{2}t + C_{3}\ \cdots\ (\ast \ast) \\
  \end{aligned}
\end{equation*}

$(\ast)$ の式に対してみると,

\begin{equation*}
  \begin{aligned}
                    &\ \dot{x} = C_{1}t + C_{2}\\
    \Leftrightarrow &\ v = at + v_{0}\\
  \end{aligned}
\end{equation*}

$(\ast \ast)$ の式に対してみると,

\begin{equation*}
  \begin{aligned}
                    &\ x = \dfrac{1}{\ 2\ }C_{1}t^{2} + C_{2}t + C_{3} \\
    \Leftrightarrow &\ x = \dfrac{1}{\ 2\ }at^{2} + v_{0}t + x_{0} \\
  \end{aligned}
\end{equation*}

これらのことから,

\begin{table}[hb]
  \centering
  \begin{tabular}{cc}
    (左辺) =
    $
    \begin{cases}
      \ddot{x} = a \\
      \dot{x} = v \\
      x = x \\
    \end{cases}
    $
    &
    $
    (右辺) =
    \begin{cases}
      C_{1} = a \\
      C_{2} = v_{0} \\
      C_{3} = x_{0}
    \end{cases}
    $
  \end{tabular}
\end{table}

と記述できるため,そもそも運動方程式を解いたらいわゆる「へんな公式」になる.

\noindent
次に $(\ast)$ と $(\ast \ast)$ を連立すると,

\begin{equation*}
  \begin{aligned}
    & \dot{x} = C_{1}t + C_{2} \\
    \Leftrightarrow&\ t = \dfrac{\ \dot{x}-C_{2}\ }{C_{1}} \\
    & x = \dfrac{1}{\ 2\ }C_{1}t^{2} + C_{2}t + C_{3} \\
    \Leftrightarrow &\ 2x = C_{1}\left(\dfrac{\ \dot{x}-C_{2}\ }{C_{1}}\right)^{2} + 2C_{2}\left(\dfrac{\ \dot{x}-C_{2}\ }{C_{1}}\right) + 2C_{3} \\
    \Leftrightarrow &\ v^{2} - v_{0}^{2} = 2C_{1}x
  \end{aligned}
\end{equation*}

唯一,$t$ が関与しない式であるが毎回連立しないといけない.

\subsection{第5回\ 問題演習}

\begin{enumerate}
  \item 質量$m=10$[kg]の小さな物体が,水平と角度20度をなす粗い斜面上に止まっている.物体に働く垂直抗力の大きさを答えよ.
  \begin{table}[hb]
    \centering
    \begin{tabular}{rcccccc}
      $\mathbb{G}$ & $=$ & ( & $mg \sin 20^{\circ}$ & ,& $-mg \cos 20^{\circ}$ & ) \\
      $\mathbb{N}$ & $=$ & ( & $0$ & ,& $N$ & ) \\
      $+\ )\ \mathbb{F}$ & $=$ & ( & $-F$ & ,& $0$ & ) \\
    \hline
      $\mathbb{G+N+F}$ & $=$ & ( & $mg \sin 20^{\circ}-F$ & ,& $N-mg \cos 20^{\circ}$ & ) \\
    \end{tabular}
  \end{table}

  斜面上に静止しているため,$\ddot{x}=0$ である.そのため,$(合力) = 0$ となる.
  題意に則すと,$\mathbb{G+N+F} = 0$ となるので以下のような連立方程式が立てられる.

  \begin{equation*}
    \begin{aligned}
      \begin{cases}
        mg \sin 20^{\circ}-F = 0 \\
        N-mg \cos 20^{\circ} = 0
      \end{cases}
      &\Leftrightarrow
      \begin{cases}
        F = mg \sin 20^{\circ} \\
        N = mg \cos 20^{\circ}
      \end{cases} \\
      &\Leftrightarrow
      \begin{cases}
        F = 10 \times 9.8 \times 0.34 \\
        N = 10 \times 9.8 \times 0.94
      \end{cases} \\
      &\Leftrightarrow
      \begin{cases}
        F = 33.32 \\
        N = 92.12
      \end{cases} \\
    \end{aligned}
  \end{equation*}

  垂直抗力は重力の反作用にあたるため,$N \fallingdotseq 92$ [N]となる.

\hs{0}

  \item[\textbf{別解?}] 重力の反作用が垂直抗力のため,$N = mg \cos 20^{\circ}$ [N] と記述できる.

\end{enumerate}

\subsection{第6回\ 問題演習}

\begin{enumerate}
  \item 地面に固定された滑らかな水平面に,質量$3.5$[kg]の物体が止まっている.この物体を,水平面に平行な一定方向に,大きさ$14$[kg \cdot m/$\mathrm{sec}^{2}$]の力で押し続けた.
  \begin{itemize}
    \begin{equation*}
      \begin{aligned}
                        &\ m \ddot{x} = F \\
        \Leftrightarrow &\ 3.5 \ddot{x}(t) = 14 \\
        \Leftrightarrow &\ \dot{x}(t) = At + B \\
        \Leftrightarrow &\ x(t) = 2t^{2} + Ct + D \\
      \end{aligned}
    \end{equation*}
    初期条件より,
    $
    \begin{cases}
      x(0)=0 \\
      \dot{x}(0)=0
    \end{cases}
    $
    となる.\\
    初期条件を考慮に入れると,
    $
    \begin{cases}
      距離\ x(t)=2t^{2} \\
      速度\ \dot{x}(t)=4t
    \end{cases}
    $ となる. 

\clearpage

    \item 押し始めてから$3.0$秒後の物体の位置と速さを答えよ.

\vs{11}
    時刻 $t=3$ より,
    $
    \begin{cases}
      距離\ x(3)=2 \times 3^{2} = 18 \\
      速度\ \dot{x}(3) = 4 \times 3 =12
    \end{cases}
    $ となる.

\vs{11}
    \therefore\ 元の位置から力の方向に,$18$[m] の位置で $12$[m/s] の速さで運動する.

\vs{11}

    \item 物体の速さが$20$[m/s]になる瞬間はいつか答えよ.
    
\vs{11}
    速さが $20$[m/s] より,
\vs{11}
    \begin{equation*}
      \begin{aligned}
                        &\ |\dot{x}(t)| = 4t \\
        \Leftrightarrow &\ 4t = 20 \\
        \Leftrightarrow &\ t = 5 \\
      \end{aligned}
    \end{equation*}

\vs{11}
    \therefore\ 押し始めてから $5$秒後に物体の速さが$20$[m/s] になる.

\vs{11}

    \item 物体が距離$32$[m]進んだ瞬間の速さを答えよ.

\vs{11}
    距離が $32$[m/s] より,
\vs{11}
    \begin{equation*}
      \begin{aligned}
                        &\ 位置\ x(t) = 32 \\
        \Leftrightarrow &\ 2t^{2} = 32 \\
        \Leftrightarrow &\ t = 4 \\
                        &\ 速さ\ |\dot{x}(4)| = 4 \times 4 \\
        \Leftrightarrow &\hs{53}= 16
      \end{aligned}
    \end{equation*}

\vs{11}
    \therefore\ 瞬間の速さが $16$[m/s] のとき,物体が $32$[m] 進んだ.

  \end{itemize}
\end{enumerate}

\subsection{第7回\ 問題演習}

\begin{enumerate}
  \item 天井に糸で吊るされた質量 $m=4$[kg] のおもりが,手で水平方向に $37^{\circ}$ 押されて,静止している,手で押している力 $\mathbb{F}$ の大きさ $F$ を数値と適切な単位で答えなさい.\\
  なお,$\sin 37^{\circ}=\dfrac{3}{5}$,$\cos 37^{\circ}=\dfrac{4}{5}$,$\tan 37^{\circ}=\dfrac{3}{4}$ とする.
  \begin{table}[hb]
    \centering
    \begin{tabular}{rcccccc}
      $\mathbb{F}$ & $=$ & ( & $F$ & ,& $0$ & ) \\
      $\mathbb{G}$ & $=$ & ( & $0$ & ,& $-mg$ & ) \\
      $+\ )\ \mathbb{T}$ & $=$ & ( & $-T \sin 37^{\circ}$ & ,& $T \cos 37^{\circ}$ & ) \\
    \hline
      $\mathbb{F+G+T}$ & $=$ & ( & $F-T \sin 37^{\circ}$ & ,& $-mg+T \cos 37^{\circ}$ & ) \\
    \end{tabular}
  \end{table}

\clearpage

題意より,$\mathbb{F+G+T} = 0$ となるので以下のような連立方程式が立てられる.

  \begin{equation*}
    \begin{aligned}
      \begin{cases}
        F-T \sin 37^{\circ} = 0 \\
        -mg+T \cos 37^{\circ} = 0
      \end{cases}
      &\Leftrightarrow
      \begin{cases}
        F-\dfrac{3}{\ 5\ }T = 0 \\
        -mg+\dfrac{4}{\ 5\ }T = 0
      \end{cases} \\
      &\Leftrightarrow
      \begin{cases}
        5F - 3T = 0 \\
        -196 + 4T = 0
      \end{cases} \\
      &\Leftrightarrow
      \begin{cases}
        T = 49 \\
        F = 29.4
      \end{cases} \\
    \end{aligned}
  \end{equation*}

  力 $\mathbb{F}$ の大きさ $F$ を求めればよいので,$F = 29.4$ [kg \cdot m/$\mathrm{s}^2$]となる.

\vs{11}
  \item 地面のある地点からある時刻に,質量 $m=200$[g] のボールを,水平から角度 $30^{\circ}$ の方向に速さ $v_{0} = 19.6$[m/s] で打ち出す.空気抵抗は無視しても良い.重力加速度は $g = 9.8$[m/$\mathrm{s}^{2}$] とする.

\vs{11}
  \begin{itemize}
    \item 座標系を設定し,ボールの運動方程式を立てなさい.\\

\vs{-11}
    運動方程式 $m \ddot{r} = -mg \mathbb{j}$ より,
    \begin{equation*}
      m(\ddot{x}(t),\ddot{y}(t))=(0,-mg)
    \end{equation*}
    \item 運動方程式を解いて,ボールの運動を求めなさい.

\vs{11}
    \begin{table}[hb]
      \centering
      \begin{tabular}{l|l}
        $x$ について,$m \ddot{x}(t) = 0$ より, & $y$ について,$m \ddot{y}(t) = -mg$ より, \\
        $
        \begin{aligned}
          &\ddot{x}(t)=0 \\
          \Leftrightarrow &\ \dot{x}(t) = a \\
          \Leftrightarrow &\ \x(t) = At+B
        \end{aligned}
        $
        &
        $
        \begin{aligned}
          &\ddot{y}(t)=-g \\
          \Leftrightarrow &\ \dot{y}(t) = -gt+C \\
          \Leftrightarrow &\ y(t) = - \dfrac{1}{\ 2\ }gt^{2}+Ct+D
        \end{aligned}
        $ \\ \\
        初期条件より, & 初期条件より,\\[11pt]
        $
        \begin{cases}
          x(0)=0 \\
          \dot{x}(0)=19.6 \cos 30^{\circ}
        \end{cases}
        $
        &
        $
        \begin{cases}
          y(0)=0 \\
          \dot{y}(0)=19.6 \sin 30^{\circ}
        \end{cases}
        $
      \end{tabular}
    \end{table}

\vs{11}
    \item ボールが地面に戻る時刻とその地点を答えなさい.
    
\vs{11}
    ボールが地面に戻る時間は,$y(t)=0$ より,
    \begin{equation*}
      0=-\dfrac{1}{2}gt^{2}+Ct+D
    \end{equation*}

\clearpage

    初期条件を考慮に入れると,
    \begin{equation*}
      \begin{aligned}
        & 0=-\dfrac{1}{2} \times 9.8t^{2}+19.6 \times \dfrac{1}{\ 2\ }t \\
        \Leftrightarrow &\ 0=-4.9t^{2}+9.8t \\
        \Leftrightarrow &\ t=2 \\
        \therefore &\ x(2)=19.6 \sqrt{3}
      \end{aligned}
    \end{equation*}

\vs{11}
    \item ボールの最高点の高さを答えなさい.
    
\vs{11}
    ボールが最高点に到達するまでの時間は,戻ってくるまでの時間の半分なので1秒となる.
    \begin{equation*}
      \begin{aligned}
        \therefore \ y(1) &= -\dfrac{1}{\ 2\ } \times 9.8 \times 1^{2} + 19.6 \times \dfrac{1}{\ 2\ } \times 1 \\
        &=\ 4.9
      \end{aligned}
    \end{equation*}
  \end{itemize}

\vs{11}
  \item 粗い水平面を運動する質量 $m$ の物体がある.面と物体との動摩擦係数は $\mu = 0.25$ である.物体の速さ $v_{0}=7.0$[m/s] で打ち出したとき,止まるまでに面を進む距離 $L$ を求めよ.\\[-10pt]
  \item[HINT] 動摩擦係数の大きさは $F=\mu N=\mu mg$ である.方向は運動と逆.進んだ距離は止まった時(速度がゼロになった時)の位置から求まる.

\vs{11}
    運動方程式 $m \ddot{x}=-\mu mg$ より,
    \begin{equation*}
      \begin{aligned}
        &\ \ddot{x}(t) = - \mu g \\
        \Leftrightarrow &\ \dot{x}(t) = - \mu gt + E \\
        \Leftrightarrow &\ x(t) = -\dfrac{1}{\ 2\ }gt^{2}+Et+F
      \end{aligned}
    \end{equation*}
    初期条件より,
    $
    \begin{cases}
      \mu = 0.25 \\
      E = 7.0 \\
      F = 0
    \end{cases}
    $
    となる.
    初期条件を考慮に入れると,
    \begin{equation*}
      \begin{aligned}
        & 0= - \mu gt + E \\
        \Leftrightarrow &\ t= \dfrac{E}{\mu g}
      \end{aligned}
    \end{equation*}
    $L = x \left(\dfrac{E}{\mu g}\right)$ となるので,
    \begin{equation*}
      \begin{aligned}
        x \left(\dfrac{E}{\mu g}\right) &= -\dfrac{1}{\ 2\ } \mu g \left(\dfrac{E}{\mu g}\right)^{2} + E \times \dfrac{E}{\mu g} \\
                                        &= -\dfrac{1}{\ 2\ } \dfrac{E^{2}}{\mu g} + \dfrac{E^{2}}{\mu g} \\
                                        &= \dfrac{1 \cdot E^{2}}{2 \cdot \mu g}
      \end{aligned}
    \end{equation*}

\clearpage

    数値を代入すると,
    \begin{equation*}
      \begin{aligned}
        L &= \dfrac{1}{\ 2\ } \times \dfrac{(7.0)^2}{0.25 \times 9.8} \\
          &= \dfrac{1}{\ 2\ } \times \dfrac{49}{2.45} \\
          &= \dfrac{49}{4.9} \\
          &= 10
      \end{aligned}
    \end{equation*}

\end{enumerate}

\pagestyle{fancy}
\lhead{\ 物体のポテンシャル}
\rhead{}
\cfoot{\thepage}
  
\section{仕事・エネルギ}

正直,運動方程式でも解けるためあまり重要視しなくても良い,追加オプション的立場である.
それよりも運動方程式を立てる方が難しいので,立てられない人はそちらを優先するべきである.

\subsection{仕事}
仕事は「変位」と呼ばれる物理量に左右される.似た言葉に「移動距離」がある.

\hs{0}

\begin{screen}
  \begin{description}
    \vs{5}
    \item[移動距離]実際に移動した距離(スカラーの加法で求められる.)\\
    \vs{-8}
    \item[変位]最初にいた地点から最終的に移動した地点間の距離(ベクトルの加法で求められる.)
    \vs{5}
  \end{description}
\end{screen}

\hs{0}

簡単に例を挙げる.

\hs{0}

\begin{screen}
  \begin{description}
    \vs{5}
    \item[Aさん] $0$[m] 地点 から $10$[m] 地点 まで移動した. \\
    \vs{-8}
    \item[Bさん] $0$[m] 地点 から $5$[m] 地点 まで移動した後,折り返して $5$[m] 移動した.
    \vs{5}
  \end{description}
\end{screen}

\hs{0}

Aさん,Bさんともに実際に動いた距離(移動距離)は $10$[m] である.
しかし,Aさんは 最初は $0$[m] 地点 であるが 最終地点は $10$[m] 地点にいる.
一方で,Bさんは 最初の地点と最終地点は同じ $0$[m] 地点にいる.
そのため,Aさんは仕事「した」が,Bさんは仕事「していない」と捉えられる.

\footnote{
  力尽きたので,解説はここまでにします.物理学概論Ⅱでは,力学を扱わないようなので更新は恐らくしないと思います.
}

\end{document}