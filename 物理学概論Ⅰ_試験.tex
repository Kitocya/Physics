\documentclass[a4paper,11pt]{ltjsarticle}
\usepackage{nomal_preamble}
\usepackage{wrapfig}

\title{物理学概論Ⅰ\ 試験}
\author{ほうじ茶}
\date{}

\begin{document}

\pagestyle{fancy}
\lhead{\ 力学}
\rhead{}
\cfoot{\thepage}

\begin{enumerate}
    \item 水平な地面に立つ高さ$h=125$[m]のビルの屋上から,
    質量$m$のボールを,速さ$v_0=3.0$[m/s]で水平かつ真東に打ち出した.
    このボールの運動を予測したい.下図のように地面に原点をとった2次元デカルト座標系を設定し,
    打ち出してからの時間を$t$[s]として,ボールの位置ベクトルをこの座標系で
    $\bm{r}(t)=(x(t),y(t))$[m]と書くことにする.以下の問に答えなさい.
    ただし,ボールに対する空気の影響は無視できる.
    問(1)と問(2)の答えには,なるべく数値を使わずに文字を使いなさい.
    問(3)と問(4)の数値の答えには,重力加速度の大きさ$g=10\ [\mathrm{m/s^2}]$を使いなさい.

    \begin{center}
    \begin{tikzpicture}
        \fill[lightgray] (0,0) rectangle (-1,3);
        \draw[->,>=stealth,very thick] (-2,0)--(3,0)node[above]{$x$}node[right]{東}; %x軸
        \draw[->,>=stealth,very thick] (0,-0.5)--(0,4)node[right]{$y$}; %y軸
        \draw (0,0)node[below left]{O}; %原点
        \draw[semithick] (0,0) rectangle (-1,3)node[left]{$h=125$[m]};
        \draw (1.5,0)node[below]{地面};
        \fill (0,3) circle (0.08);
        \draw[->,semithick] (0,3)--(0.8,3)node[above right]{$v_0=3.0$[m/s]};
        \draw (1,2.7) circle (0.08);
        \draw (1.5,2.4) circle (0.08)node[right]{$(x(t),y(t))$};
        \draw (2,1.7) circle (0.08);
    \end{tikzpicture}
    \end{center}
    
    \begin{enumerate}[label=(\arabic*)]
        \item 問題文で用意された座標系と函数を使って,打ち出されてから地面に着くまでの間のボールの\textbf{運動方程式}と
        ボールの\textbf{初期条件}を書きなさい.

        \item 問1の方程式と条件を解いて,空中に飛んでいるボールの\textbf{運動},つまり,函数$x(t)$と$y(t)$を答えなさい.
    \end{enumerate}
\end{enumerate}



\end{document}