\documentclass[a4paper,11pt]{ltjsarticle}
\usepackage{nomal_preamble}
\usepackage{anyfontsize}

\title{物理学概論Ⅱ\ 解説}
\author{ほうじ茶}
\date{}

\begin{document}

\pagestyle{fancy}
\lhead{\ 熱力学}
\rhead{}
\cfoot{\thepage}

\maketitle
\tableofcontents
\clearpage

\section{熱力学}

熱や仕事のやり取りによる物質の状態変化を扱う.
ただ,原子や分子のような粒子は知られてない中で完成された学問であるため,
1つ1つの分子の運動を考える(量子力学)の考え方ではなく,巨視的視点から考える必要がある.

化学概論Ⅰで失望した人が多いと思われるが,量子力学の元は熱力学であるので似た話が出てくるが
あきらめないように!

\subsection{パラメータ}


\begin{itemize}
    \item 圧力$p$ [Pa]
    \item 体積$V$ [$\mathrm{m^3}$]
    \item 温度$T$ [K](\ $T\ [\mathrm{K}] = 273 + t\ [\mathrm{{}^\circ C}]$\ )
\end{itemize}

この3つのパラメータは2つ決まれば,残りの1つが一意に定まる.

\subsection{ボイル・シャルルの法則}

\begin{description}
    \item[ボイルの法則] 温度$T$を一定にすると,体積$V$と圧力$p$が反比例の関係になる.
    \begin{equation*}
        {\color{cyan} pV = \mathrm{const.}(T:一定)}
    \end{equation*}
    \item[シャルルの法則] 圧力$p$を一定にすると,体積$V$と温度$T$が比例の関係になる.
    \begin{equation*}
        {\color{cyan} \dfrac{V}{T} = \mathrm{const.}(p:一定)}
    \end{equation*}
    \item[ボイル・シャルルの法則] 上記2式を1つにまとめると,以下のようになる.
    \begin{equation*}
        {\color{cyan} \dfrac{pV}{T} = \mathrm{const.}(R:一定)}
    \end{equation*}
\end{description}

ボイル・シャルルの法則を記述したが,次で扱う理想気体の状態方程式で包含されるため覚える必要はない.

\subsection{理想気体}

分子間力がなく,分子の体積が0の気体のこと.理想気体の状態方程式は,

\begin{equation*}
    {\color{cyan} pV = nRT}
\end{equation*}

\begin{itemize}
    \item 物質量$n$ [mol]
    \item 気体定数$R \fallingdotseq 8.31$ [J/(mol・K)]
\end{itemize}

\pagestyle{fancy}
\lhead{\ 電磁気}
\rhead{}
\cfoot{\thepage}

\section{電気}

\end{document}