\documentclass[a4paper,11pt]{ltjsarticle}
\usepackage{nomal_preamble}

\title{物理学概論Ⅱ\ 解説}
\author{ほうじ茶}
\date{}

\begin{document}

\pagestyle{fancy}
\lhead{\ 熱力学}
\rhead{}
\cfoot{\thepage}

\maketitle
\tableofcontents
\clearpage

\section{熱力学}

熱や仕事のやり取りによる物質の状態変化を扱う.
ただ,原子や分子のような粒子は知られてない中で完成された学問であるため,
1つ1つの分子の運動を考える(量子力学)の考え方ではなく,巨視的視点から考える必要がある.

化学概論Ⅰで失望した人が多いと思われるが,量子力学の元は熱力学であるので似た話が出てくるが
あきらめないように!

\subsection{パラメータ}


\begin{itemize}
    \item 圧力$p$ [Pa]
    \item 体積$V$ [$\mathrm{m^3}$]
    \item 温度$T$ [K](\ $T\ [\mathrm{K}] = 273 + t\ [\mathrm{{}^\circ C}]$\ )
\end{itemize}

この3つのパラメータは2つ決まれば,残りの1つが一意に定まる.

\subsection{ボイル・シャルルの法則}

\begin{description}
    \item[ボイルの法則] 温度$T$を一定にすると,体積$V$と圧力$p$が反比例の関係になる.
    \begin{equation*}
        {\color{cyan} pV = \mathrm{const.}(T:一定)}
    \end{equation*}
    \item[シャルルの法則] 圧力$p$を一定にすると,体積$V$と温度$T$が比例の関係になる.
    \begin{equation*}
        {\color{cyan} \dfrac{V}{T} = \mathrm{const.}(p:一定)}
    \end{equation*}
    \item[ボイル・シャルルの法則] 上記2式を1つにまとめると,以下のようになる.
    \begin{equation*}
        {\color{cyan} \dfrac{pV}{T} = \mathrm{const.}(R:一定)}
    \end{equation*}
\end{description}

ボイル・シャルルの法則を記述したが,次で扱う理想気体の状態方程式で包含されるため覚える必要はない.

\subsection{理想気体}

分子間力がなく,分子の体積が0の気体のこと.理想気体の状態方程式は,

\begin{equation*}
    {\color{cyan} pV = nRT}
\end{equation*}

\begin{itemize}
    \item 物質量$n$ [mol]
    \item 気体定数$R \fallingdotseq 8.31$ [J/(mol・K)]
\end{itemize}

\clearpage

\subsection{気体分子運動論}

上記で述べた理想気体の状態方程式は,化学的かつ経験則で求められた.
しかしドルトンが原子というものを提唱し,ミクロな世界でも適応できる力学を考えるということで量子力学というものができた.
この気体分子運動論は熱力学という「巨視的」に見てきたものを,原子の運動すなわち「微視的」に見ていくという
化学と物理との関係性を表した式である.

\subsubsection*{設定}

\begin{itemize}
    \item 1つの原子が2つの壁に挟まれている.
    \item 原子は壁からもう一方の壁に衝突する(ベクトル).
    \item 2つの壁の距離は$\ell$,
    \item 原子の速度は$u_x$\ とする.
\end{itemize}

1秒間に向かいの壁に衝突する回数は,

\begin{equation*}
    {\color{cyan} \dfrac{u_x}{\ell}\ \mathrm{[Hz/s]}}
\end{equation*}

粒子が衝突することで生じる運動変化量は,

\begin{equation*}
    {\color{cyan} mu_x \times \dfrac{u_x}{\ell}\ \mathrm{[Hz/s]} = \dfrac{mu_x^2}{\ell}\ \mathrm{[/s]}}
\end{equation*}

\begin{itemize}
    \item 質量$m$
\end{itemize}

今までは$u_x$の1つのベクトルを代表して求めていたが,原子の運動は3次元ベクトルのため,

\begin{equation*}
    u^2 = u_x^2 + u_y^2 +u_z^2
\end{equation*}
\begin{equation*}
    \overline{u}^2 = \overline{u_x}^2 + \overline{u_y}^2 + \overline{u_z}^2 \\
\end{equation*}

統計的に以下のように計算できる.

\begin{equation*}
    \overline{u_x}^2 = \overline{u_y}^2 = \overline{u_z}^2
\end{equation*}
\begin{equation*}
    \overline{u}^2 = 3\overline{u_x}^2
\end{equation*}

1つの分子が1つの壁に及ぼす力は.

\begin{equation*}
    {\color{cyan} \dfrac{力}{面積} = \dfrac{\tfrac{mu_x^2}{\ell}}{\ell^2} = \dfrac{mu_x^2}{\ell^3}}
\end{equation*}

\clearpage

上記のことを踏まえ1つの壁に衝突するときに生じる圧力は,

\begin{equation*}
    p = N_0 \dfrac{m \overline{u_x}^2}{V} = N_0 \dfrac{m \overline{u}^2}{3V}
\end{equation*}

\begin{itemize}
    \item 分子の個数$N_n$
\end{itemize}

ボイルの法則を表現できそうなので$pV=$を目指して式変形すると,

\begin{equation*}
    pV = \dfrac{1}{3} N_0 m \overline{u}^2
\end{equation*}

右辺が運動エネルギーを表現できそうなので$\dfrac{1}{2}u \overline{u}^2$を目指して式変形すると,

\begin{equation*}
    {\color{cyan} \dfrac{1}{3} nN_A m \overline{u}^2 = \dfrac{2}{3} nN_A \dfrac{1}{2} m \overline{u}^2}
\end{equation*}

\begin{itemize}
    \item アボガドロ数$N_A$
\end{itemize}

以上のことから以下のようにまとめられる.

\begin{equation*}
    {\color{cyan} \dfrac{1}{2} m \overline{u}^2 = \dfrac{3}{2} \dfrac{R}{N_A}T =\dfrac{3}{2}kT}
\end{equation*}

\begin{itemize}
    \item ボルツマン定数$k$
\end{itemize}

\subsection{熱力学第1法則}

具体的な式を以下に記述する.

\begin{equation*}
    {\color{cyan} _\Delta Q = _\Delta U + _\Delta W}
\end{equation*}

\begin{itemize}
    \item 加えた熱量$_\Delta Q$
    \item 体積変化$_\Delta W$(仕事)
    \item 温度変化$_\Delta U$(内部エネルギー)
\end{itemize}

シリンダー内の空気の流れを考える.そのシリンダーに熱を加えると,
$
\begin{cases}
    気体は{\color{cyan}膨張}する.\\
    温度は{\color{cyan}低下}する.
\end{cases}
$

この系を押さえておけば,何となくのイメージングができる.

\clearpage

\subsection{熱力学第2法則}

これは,いろいろな法則をまとめたものである.1つにまとめて以下に記述する.

\begin{equation*}
    {\color{cyan}熱源から熱を受け取り,それを全て仕事に変えるような熱サイクルは存在しない.}
\end{equation*}

上記に記述したのは,トムソンの原理である.他の原理についても列挙する.

\begin{description}
    \item[クラウジウスの原理]\mbox{}\\ 低温熱浴から高温熱浴に熱を移して,他に影響を残さないような熱サイクルは存在しない.
    \item[オストヴァルトの原理]\mbox{}\\  第二種永久機関(熱源から熱を吸収し,ずっと動き続ける熱サイクルのこと)は存在しない.
\end{description}

基本的に,同じ主張をしていることが分かる.これらをまとめて,カルノーの定理を導ける.

\begin{description}
    \item[カルノーの定理]\mbox{}\\
    \begin{equation*}
        \eta^{\prime\prime} \leq \eta = \eta^{\prime}
    \end{equation*}
\end{description}

任意の物質を作業物質として動かした可逆な熱機関の熱効率$\eta^{\prime}$は,
理想気体を作業物質として動かしたカルノーサイクルの熱効率$\eta$に等しい.
また,不可逆な熱機関の熱効率$\eta^{\prime\prime}$は,上記の式を満たす.

\subsection{熱力学第3法則}

具体的な式を以下に記述する.

\begin{equation*}
    \lim_{T → 0}S(T)=0
\end{equation*}

純物質の完全結晶のエントロピーが絶対零度でゼロとなる法則のことである.
\footnote{物理学では出てこないので割愛する.}

\end{document}